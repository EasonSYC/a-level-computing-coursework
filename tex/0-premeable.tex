\usepackage[a4paper,margin=1in]{geometry}
\usepackage{graphicx}
% \usepackage[utf8]{inputenc}
\usepackage[british]{babel}
\usepackage[backend=biber,sortcites,sorting=nty,giveninits=true]{biblatex}

\usepackage{fancyhdr}
\usepackage{lastpage}
\usepackage{appendix}

\usepackage[AutoFakeSlant=true]{xeCJK}
\setCJKmainfont{YuMincho}
\setCJKsansfont{YuGothic}
\setCJKmonofont{YuGothic}

\usepackage[format=plain,justification=centering]{caption}

\usepackage{amsfonts}
\usepackage{amsmath}

\usepackage[dvipsnames]{xcolor}
\usepackage{fontawesome5}
\usepackage{amssymb}
\usepackage{xltxtra}
\usepackage{textcomp}

\usepackage{tocloft}

\usepackage[section]{algorithm}
\usepackage[noEnd=false,spaceRequire=false,rightComments=false]{algpseudocodex}

\newlistof{listing}{lol}{List of Listings}
\usepackage[section,newfloat]{minted}
\usepackage{csquotes}

\usepackage{chngcntr}
\usepackage{hyperref}

\SetupFloatingEnvironment{listing}{name={Listing},fileext=lol}

% https://tex.stackexchange.com/a/131585/312350
\makeatletter
\begingroup
\let\newcounter\@gobble
\let\setcounter\@gobbletwo
\globaldefs\@ne
\let\c@loadepth\@ne
\newlistof{algorithms}{loa}{\listalgorithmname}
\endgroup
\let\l@algorithm\l@algorithms
\makeatother

\newcommand{\NumberWidth}{4em}
\setlength{\cftfignumwidth}{\NumberWidth}
\setlength{\cfttabnumwidth}{\NumberWidth}
\setlength{\cftlistingnumwidth}{\NumberWidth}
\setlength{\cftalgorithmsnumwidth}{\NumberWidth}

\newcommand{\IndentWidth}{1.5em}
\setlength{\cftlistingindent}{\IndentWidth}
\setlength{\cftalgorithmsindent}{\IndentWidth}

\newcommand{\algorithmautorefname}{Algorithm}

\counterwithin{figure}{section}
\counterwithin{table}{section}

\addbibresource{ref/microsoft-docs.bib}
\addbibresource{ref/japan-meteorology.bib}
\addbibresource{ref/mapsui-avalonia.bib}
\addbibresource{ref/others.bib}

\setcounter{secnumdepth}{3}

\newcommand{\CodeBasePath}{early-earthquake-and-tsunami-warning-viewer/}

\newenvironment{normallisting}{\captionsetup{type=listing}}{\vspace{\floatsep}}
\newcommand{\InputLine}[4]{
    \inputminted[firstline=#3, lastline=#4]{#1}{#2}
}

\makeatletter
\newcommand{\InputLines}[2]{
    \InputLines@i{#1}{#2}
}
\newcommand{\InputLines@i}[2]{
    \@ifnextchar\bgroup
    {\InputLines@ii{#1}{#2}}
    {}
}
\newcommand{\InputLines@ii}[4]{
    \InputLine{#1}{#2}{#3}{#4}
    \InputLines@i{#1}{#2}
}

\makeatother

\newmintinline[Code]{text}{}
\newcommand{\Colour}[1]{\Code{#1}\space\textcolor[HTML]{#1}{\(\blacksquare\)}}

\graphicspath{{./fig/}{\CodeBasePath python/}}

\setminted{
    autogobble,
    breaklines,
    breakanywhere,
    linenos,
    mathescape,
    style=sas
}

\newcommand{\ToDo}[1]{\textbf{\color{red}{\large{To-Do: #1}}}}

\newcommand{\Title}{Early Earthquake and Tsunami Warning Viewer}
\newcommand{\SubTitle}{AQA A-Level Computing NEA Report}
\newcommand{\Author}{Yicheng Shao (Eason)}
\newcommand{\CandNo}{7616}
\newcommand{\Date}{\today}

\title{\Title}
\author{\Author}
\date{\Date}

\newcommand{\SetFooter}{
    \lfoot{Candidate \textnumero: \CandNo}
    \rfoot{Page \thepage\space of \pageref*{LastPage}}
}

\newcommand{\GitHubHref}[2]{
    \href{https://github.com/#1/#2}{\faGithub\space #1/#2}
}

\newcommand{\SetMainStyle}{
    \fancyhf{}
    \lhead{\Author}
    \rhead{\Title}
    \SetFooter
}

\fancypagestyle{plain}{
    \renewcommand{\headrulewidth}{0pt}
    \fancyhf{}
    \SetFooter
}