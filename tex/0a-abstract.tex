This application displays early earthquake warnings and tsunami warnings happening in Japan to the user in real time. A map is coloured for the user to view the predicted intensities or tsunami warning level, and the epicentre of the earthquake is plotted, as well as the real-time wavefronts (P/S). The map also includes a layer which displays the real-time seismic observations at observation stations, and the user is able to customise this layer to switch between different types of observation stations and different data.

In addition to this, users are able to view a list of past earthquakes, and select an earthquake to view the details, including colouring the map for the observed intensity of observation stations, the hypocentre position, and the maximum intensities by regions. An expandable intensity tree on the sidebar to shows the intensities of observation stations in hierarchy order of intensity and then the geographical location.

The user also have access to a settings page, where they can authorise the application by API Key or OAuth 2. They would also be able to connect/disconnect from WebSocket, and view a list of current connections. Customisation features involve the real-time observation layer and the language of the application.

To acquire the information, the DM-D.S.S. information source is used, involving API Calls, authentication (including OAuth), and WebSocket connections. Data are in JSON format so deserialisation is involved. To plot the wavefront accurately, the official data source from the Japan Meteorological Agency is used, and linear search and interpolation algorithm is executed. Data is stored in a file, so file I/O and format checking using regular expressions is involved. The Kyoshin Monitor data source from the NIED is used for the real-time seismic observation data, and colours will be extracted from the image, then converted to a normalised value which involves a binary search algorithm to calculate an inverse cubic. An algorithm is also designed to detect shake events using the data acquired from the Kyoshin Monitor.

Object-oriented programming is the primary paradigm used in this (though functional programming-like list manipulations are involved), and several programming patterns including factory pattern and adaptor pattern are used to ensure the maintainability. There is a significant use of OOP relationships such as interfaces for abstraction and inversion of control, inheritance for code reuse, and composition and aggregation. There is a significant use of asynchronous programming, including the use of cancellation tokens and task factories. Events are used to notify other components of something of interest. Logging and exception handling is used significantly in the application for maintainability and robustness.

In this report, the analysis of the problem, including an interview of a potential user and reviewing of existing applications, was conducted. The requirements were then specified, including a critical path for the approach, and the colour scheme and display details are determined. The design follows, with a decomposition of the problem and the analysis of the external data sources, and then the algorithms are designed, with flowcharts or pseudocode available to describe the process. The UI is designed with mock-ups and explanation attached. OOP patterns are introduced, and the classes are designed, involving the use of the design patterns above, and class diagrams are involved. The modelling of DTOs differs slightly from the official data source, which improves code reusability.

The application was then coded, and the application was tested against the specification in the analysis section, and a testing video is attached. Evaluation is finally conducted on the application, and feedback was received from the potential end user, which are then compiled improvements suggestions for the application.

The code is written in C\# using .NET 9.0 framework, using Avalonia as the UI, \Code{CommunityToolkit.Mvvm} for the MVVM pattern, and Mapsui for the display of the map.