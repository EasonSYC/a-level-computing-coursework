\chapter{Testing}
\section{Test Strategy}

The primary testing method for the application will be integration testing, due to the time limitation of this project, and the great amount of components involved. An integrated test showing the application working also probably will be the best way to demonstrate the application is working as expected, and that all components are working together (which probably means all components are working individually as well).

However, the components have been designed with unit testing in mind, since they all use interfaces when they have dependencies, which allows for mock-up classes to be used in place of the real classes.

The test is split into four parts to be tested:
\begin{itemize}
    \item The settings page (WebSockets, authentication and language settings);
    \item The past-earthquake page;
    \item The real-time page (with mock-up EEWs and Tsunami Warnings), including customisation for Kyoshin Monitor, and the window-switching functionality;
    \item Robustness testing (e.g. unstable internet connection, defensive programming against user input).
\end{itemize}

The tests for the table section are outlined in \autoref{tab:tests-settings-auth} for authentication, \autoref{tab:tests-settings-websocket} for WebSocket, and \autoref{tab:tests-settings-language} for language settings.

\begin{table}[htp]
    \centering
    \begin{tabular}{c|l|l}
        \textnumero & Test Description                     & Expected Outcome                         \\
        \hline
        1a(i)       & Set the API Key                      & API Key set                              \\
        1a(ii)      & Change the contents of the input box & API Key unset                            \\
        1a(iii)     & Close and relaunch the App           & Application not authenticated            \\
        1a(iv)      & Press the OAuth Button               & Browser page launched for authentication \\
        1a(v)       & Close and relaunch the App           & Authentication status persists           \\
        1a(vi)      & Attempts to set API Key              & Set button disabled                      \\
        1a(vii)     & Press the OAuth Button               & Application disconnects from OAuth       \\
        1a(viii)    & Close and relaunch the App           & Application not authenticated            \\
        1a(ix)      & Set the API Key                      & API Key set                              \\
        1a(x)       & Close and relaunch the App           & Authentication status persists
    \end{tabular}
    \caption{Tests for the Settings Page (Authentication)}
    \label{tab:tests-settings-auth}
\end{table}

\begin{table}[htp]
    \centering
    \begin{tabular}{c|l|l}
        \textnumero & Test Description                        & Expected Outcome                             \\
        \hline
        1b(i)       & Press WebSocket button                  & WebSocket connected                          \\
        1b(ii)      & Press refresh WebSocket List Button     & List displayed, with empty connections       \\
        1b(iii)     & Press disconnect button for current App & WebSocket disconnected                       \\
        1b(iv)      & Press refresh WebSocket List Button     & Disconnected WebSocket connection disappears \\
        1b(v)       & Press disconnect button for another App & WebSocket disconnected                       \\
        1b(vi)      & Press refresh WebSocket List Button     & Disconnected WebSocket connection disappears \\
    \end{tabular}
    \caption{Tests for the Settings Page (WebSocket)}
    \label{tab:tests-settings-websocket}
\end{table}

\begin{table}[htp]
    \centering
    \begin{tabular}{c|l|l}
        \textnumero & Test Description             & Expected Outcome                                          \\
        \hline
        1c(i)       & Set the language to English  & Most of the App in English                                \\
        1c(ii)      & Close and relaunch the App   & Language settings persists, correct language in whole App \\
        1c(iii)     & Set the language to Japanese & Most of the App in Japanese                               \\
        1c(iv)      & Close and relaunch the App   & Language settings persists, correct language in whole App \\
        1c(v)       & Set the language to Chinese  & Most of the App in Chinese                                \\
        1c(vi)      & Close and relaunch the App   & Language settings persists, correct language in whole App
    \end{tabular}
    \caption{Tests for the Settings Page (Language)}
    \label{tab:tests-settings-language}
\end{table}

The tests for the past-earthquake page are outlined in \autoref{tab:tests-past-earthquake}.

\begin{table}[htp]
    \centering
    \begin{tabular}{c|l|l}
        \textnumero & Test Description                   & Expected Outcome                                   \\
        \hline
        2(i)        & Press load earthquakes             & List loaded and displayed correctly                \\
        2(ii)       & Press load extra earthquakes       & More earthquakes loaded and appended to the list   \\
        2(iii)      & Press load earthquakes             & List re-loaded (with the extra earthquake removed) \\
        2(iv)       & Choose a specific earthquake       & Details displayed on side panel                    \\
        2(v)        & (None)                             & Map with hypocentre on top layer                   \\
        2(vi)       & (None)                             & Map with observation stations on next layer        \\
        2(vii)      & (None)                             & Map with regions on bottom                         \\
        2(viii)     & (None)                             & Map zoomed to regional intensities                 \\
        2(ix)       & (None)                             & Side panel details displayed and formatted         \\
        2(x)        & (None)                             & Intensity tree displayed and formatted             \\
        2(xi)       & Press view Yahoo earthquake button & Webpage to earthquake launched                     \\
        2(xii)      & Another earthquake chosen          & Application behaves as in 2(v) to 2(xi)
    \end{tabular}
    \caption{Tests for the Past Earthquake Page}
    \label{tab:tests-past-earthquake}
\end{table}

The tests for the real-time page is outlined in \autoref{tab:tests-realtime-kmoni} for Kyoshin Monitor, \autoref{tab:tests-realtime-eew} for EEWs, \autoref{tab:tests-realtime-tsunami} for Tsunami Warnings.

\begin{table}[htp]
    \centering
    \begin{tabular}{c|l|l}
        \textnumero & Test Description                           & Expected Outcome                         \\
        \hline
        3a(i)       & (None)                                     & Current time refreshing                  \\
        3a(ii)      & (None)                                     & Kyoshin Monitor data is displayed on map \\
        3a(iii)     & (None)                                     & Scale of colour is displayed correctly   \\
        3a(iv)      & (None)                                     & Quantity on scale is displayed correctly \\
        3a(v)       & (None)                                     & Data of scale is displayed correctly     \\
        3a(vi)      & Change stations settings to borehole       & Less observation station displayed       \\
        3a(vii)     & Relaunch application                       & Settings Persist                         \\
        3a(viii)    & Change stations settings to surface        & More observation station displayed       \\
        3a(ix)      & Relaunch application                       & Settings Persist                         \\
        3a(x)       & Change data settings to PGA                & Data and scale is displayed correctly    \\
        3a(xi)      & Relaunch application                       & Settings Persist                         \\
        3a(xii)     & Change data settings to Measured Intensity & Data and scale is displayed correctly    \\
        3a(xiii)    & Relaunch application                       & Settings Persist
    \end{tabular}
    \caption{Tests for the Real-time Page (Kyoshin Monitor)}
    \label{tab:tests-realtime-kmoni}
\end{table}

\begin{table}[htp]
    \centering
    \begin{tabular}{c|l|l}
        \textnumero & Test Description                           & Expected Outcome                      \\
        \hline
        3b(i)       & (Feed First Mock EEW)                      & EEW details is displayed on sidebar   \\
        3b(ii)      & (None)                                     & Window is switched to real-time page  \\
        3b(iii)     & (None)                                     & Hypocentre marked                     \\
        3b(iv)      & (None)                                     & Wavefronts drawing and updating       \\
        3b(v)       & (None)                                     & Predicted maximum intensity displayed \\
        3b(vi)      & (Feed Second Mock EEW, assumed hypocentre) & EEW displayed on sidebar              \\
        3b(vii)     & (None)                                     & Hypocentre marked (circular symbol)   \\
        3b(viii)    & (None)                                     & Wavefronts not displayed              \\
        3b(ix)      & (None)                                     & Predicted maximum intensity displayed \\
        3b(x)       & (None)                                     & Details panel switching between EEWs  \\
        3b(xi)      & (Feed Old First Mock EEW)                  & EEW Information not Updated           \\
        3b(xii)     & (Feed Updated First Mock EEW)              & EEW Behaving as in 3b(ii) to 3b(iv)   \\
        3b(xiii)    & (None)                                     & EEW removed after time
    \end{tabular}
    \caption{Tests for the Real-time Page (EEWs)}
    \label{tab:tests-realtime-eew}
\end{table}

\begin{table}[htp]
    \centering
    \begin{tabular}{c|l|l}
        \textnumero & Test Description            & Expected Outcome                      \\
        \hline
        3c(i)       & (Feed Mock Tsunami Warning) & Tsunami Warning is displayed on panel \\
        3c(ii)      & (None)                      & Window is switched to real-time page  \\
        3c(iii)     & (None)                      & Shorelines coloured and displayed     \\
        3c(iv)      & (None)                      & Tsunami Warning details displayed
    \end{tabular}
    \caption{Tests for the Real-time Page (Tsunamis)}
    \label{tab:tests-realtime-tsunami}
\end{table}

The tests for robustness testing are outlined in \autoref{tab:tests-robustness}.

\begin{table}[htp]
    \centering
    \begin{tabular}{c|l|l}
        \textnumero & Test Description                         & Expected Outcome                   \\
        \hline
        4(i)        & Press buttons when Unauthorised          & Application does not crash         \\
        4(ii)       & Input invalid API Key Format             & Button to Set disabled             \\
        4(iii)      & Set Incorrect API Key                    & Authorisation automatically resets \\
        4(iv)       & Set Invalid OAuth 2 Refresh Token        & Authorisation automatically resets \\
        4(v)        & Unstable Internet Connection             & Application does not crash         \\
        4(vi)       & Clicking buttons at rapid pace           & Application does not crash         \\
        4(vii)      & Loading earthquakes with special details & Correct text displayed
    \end{tabular}
    \caption{Tests for Robustness}
    \label{tab:tests-robustness}
\end{table}

\section{Testing Code for EEWs and Tsunami Warnings}

It is unlikely that an EEW warning will happen to be issued during this testing status. Therefore, mock EEWs and WebSockets are fed to the view model via the event handler (which is made internal for testing purposes). The data used for the mock EEWs and tsunami warnings are from the official samples of DM-D.S.S. \autocite{dmdata-sample}, modified slightly for the demonstration of the tests. The JSON is slightly altered, especially with the time and the serial number of the EEWs, to ensure that the EEWs are displayed, and the wavefronts are plotted correctly. Specifically, the telegram \href{https://sample.dmdata.jp/conversion/json/schema/eew-information/vxse45_rjtd_20110311144810.json}{here} (a mock-up of a 2011 earthquake) is used as the test data for the first mock EEW, and the telegram \href{https://sample.dmdata.jp/conversion/json/schema/eew-information/vxse44_rjtd_20180906031016.json}{here} (a mock-up of a 2018 earthquake) is used as the test data for the second mock EEW.

The serial numbers, the \Code{pressDateTime} (used for determining whether an earthquake has expired) and the \Code{originTime} (used for determining the time for the earthquake for plotting wavefronts), and \Code{isLastInfo}, \Code{isCanceled} and \Code{isWarning} (used for determining the type of the EEW) are modified to ensure the test purposes.

For the tsunami warnings, \href{https://sample.dmdata.jp/conversion/json/schema/tsunami-information/vtse41_jpos_20110311144959.json}{this telegram} is used as the test data for the tsunami warning, and the time is modified to ensure that the tsunami warning is deemed as valid by the application.

The code used to feed the tsunami warnings are listed in ...... The JSON Files are not included, since they are mostly identical with the files in the links provided above, and any differences are demonstrated in the testing videos as well.

\begin{normallisting}
    \InputLines{csharp}{\CodeBasePath EasonEetwViewer/EasonEetwViewer/App.axaml.cs}{33}{34}{44}{45}{129}{173}{193}{193}
    \caption[Code for testing within \Code{App.axaml.cs} Class]{
        Code for testing within \Code{App.axaml.cs} Class \\
        Full Code at \autoref{code-listing:EasonEetwViewer/App.axaml.cs}}
    \label{code:testing-code}
\end{normallisting}

\section{Testing Video}

The videos for testing is available on Google Drive \href{https://drive.google.com/drive/folders/1ZcaxVQKPGmVMIAyqow7Lvkem4XKcVg2m?usp=share_link}{here}. The times for the specific tests are specified in \autoref{tab:tests-time}. Note that the time displayed is when the details are attempted to be highlighted in the video -- those features might have already appeared on the screen when the previous action had been taken.

\begin{table}[htp]
    \centering
    \begin{tabular}{cc}
        \textnumero & Time in Video \\
        \hline
        1a(i)       & 00:06         \\
        1a(ii)      & 00:15         \\
        1a(iii)     & 00:22         \\
        1a(iv)      & 00:32         \\
        1a(v)       & 00:46         \\
        1a(vi)      & 01:02         \\
        1a(vii)     & 01:09         \\
        1a(viii)    & 01:13         \\
        1a(ix)      & 01:20         \\
        1a(x)       & 01:27         \\
                    &               \\
                    &
    \end{tabular}
    \begin{tabular}{cc}
        \textnumero & Time in Video \\
        \hline
        1b(i)       & 00:08         \\
        1b(ii)      & 00:12         \\
        1b(iii)     & 00:19         \\
        1b(iv)      & 00:27         \\
        1b(v)       & 00:32         \\
        1b(vi)      & 00:41         \\
        \hline
        1c(i)       & 00:10         \\
        1c(ii)      & 00:26         \\
        1c(iii)     & 00:29         \\
        1c(iv)      & 00:43         \\
        1c(v)       & 00:47         \\
        1c(vi)      & 01:01
    \end{tabular}
    \begin{tabular}{cc}
        \textnumero & Time in Video \\
        \hline
        2(i)        & 00:10         \\
        2(ii)       & 00:19         \\
        2(iii)      & 00:27         \\
        2(iv)       & 00:33         \\
        2(v)        & 00:38         \\
        2(vi)       & 00:38         \\
        2(vii)      & 00:38         \\
        2(viii)     & 00:34         \\
        2(ix)       & 00:35         \\
        2(x)        & 00:45         \\
        2(xi)       & 01:09         \\
        2(xii)      & 01:25
    \end{tabular}
    \begin{tabular}{cc}
        \textnumero & Time in Video \\
        \hline
        3a(i)       & 00:08         \\
        3a(ii)      & 00:13         \\
        3a(iii)     & 00:20         \\
        3a(iv)      & 00:23         \\
        3a(v)       & 00:26         \\
        3a(vi)      & 00:32         \\
        3a(vii)     & 00:45         \\
        3a(viii)    & 00:54         \\
        3a(ix)      & 01:09         \\
        3a(x)       & 01:16         \\
        3a(xi)      & 01:36         \\
        3a(xii)     & 01:40         \\
        3a(xiii)    & 01:54
    \end{tabular}
    \begin{tabular}{cc}
        \textnumero & Time in Video \\
        \hline
        3b(i)       & 00:46         \\
        3b(ii)      & 00:42         \\
        3b(iii)     & 00:42         \\
        3b(iv)      & 00:46         \\
        3b(v)       & 00:46         \\
        3b(vi)      & 00:55         \\
        3b(vii)     & 00:54         \\
        3b(viii)    & 00:55         \\
        3b(ix)      & 00:55         \\
        3b(x)       & 00:55         \\
        3b(xi)      & 01:05         \\
        3b(xii)     & 01:13         \\
        3b(xiii)    & 03:43
    \end{tabular}
    \begin{tabular}{cc}
        \textnumero & Time in Video \\
        \hline
        3c(i)       & 00:36         \\
        3c(ii)      & 00:36         \\
        3c(iii)     & 00:59         \\
        3c(iv)      & 00:42         \\
        \hline
        4(i)        & 00:07         \\
        4(ii)       & 00:20         \\
        4(iii)      & 00:33         \\
        4(iv)       & 01:08         \\
        4(v)        & 01:17         \\
        4(vi)       & 01:50         \\
        4(vii)      & 02:28         \\
                    &               \\
                    &
    \end{tabular}
    \caption{Times for Tests in Videos}
    \label{tab:tests-time}
\end{table}

\section{Testing against Original Requirements' Specification}
Back in the analysis section, requirements are specified in \autoref{tab:requirements-part-one}, \autoref{tab:requirements-part-two-a}, \autoref{tab:requirements-part-two-b}, and \autoref{tab:requirements-part-four-five} for different functionalities.

At some parts, the functionality is not directly demonstrated by a direct test (e.g. unit test or modular test), but the integrated test shows some of that the functionality must be working as expected, especially for DM-D.S.S. functionalities. This applies especially to \autoref{tab:requirements-part-one} in requirement 1(ii) to 1(x), and in \autoref{tab:requirements-part-two-a} in requirement 2a(v), in \autoref{tab:requirements-part-four-five} in requirement 4(vi).

The requirements are evaluated and discussed in the next section about evaluation.