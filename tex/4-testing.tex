\chapter{Testing}
\section{Test Strategy}

The primary testing method for the application will be integration testing, due to the time limitation of this project, and the great amount of components involved. An integrated test showing the application working also probably will be the best way to demonstrate the application is working as expected, and that all components are working together (which probably means all components are working individually as well).

However, the components have been designed with unit testing in mind, since they all use interfaces when they have dependencies, which allows for mock-up classes to be used in place of the real classes.

The test is split into four parts to be tested:
\begin{itemize}
    \item The settings page (WebSockets, authentication and language settings);
    \item The past-earthquake page;
    \item The real-time page (with mock-up EEWs and Tsunami Warnings), including customisation for Kyoshin Monitor, and the window-switching functionality;
    \item Robustness testing (e.g. unstable internet connection, defensive programming against user input).
\end{itemize}

The tests for the table section are outlined in \autoref{tab:tests-settings-auth} for authentication, \autoref{tab:tests-settings-websocket} for WebSocket, and \autoref{tab:tests-settings-language} for language settings.

\begin{table}[htp]
    \centering
    \begin{tabular}{c|l|l}
        \textnumero & Test Description                     & Expected Outcome                         \\
        \hline
        1a(i)       & Set the API Key                      & API Key set                              \\
        1a(ii)      & Change the contents of the input box & API Key unset                            \\
        1a(iii)     & Close and relaunch the App           & Application not authenticated            \\
        1a(iv)      & Press the OAuth Button               & Browser page launched for authentication \\
        1a(v)       & Close and relaunch the App           & Authentication status persists           \\
        1a(vi)      & Attempts to set API Key              & Set button disabled                      \\
        1a(vii)     & Press the OAuth Button               & Application disconnects from OAuth       \\
        1a(viii)    & Close and relaunch the App           & Application not authenticated            \\
        1a(ix)      & Set the API Key                      & API Key set                              \\
        1a(x)       & Close and relaunch the App           & Authentication status persists
    \end{tabular}
    \caption{Tests for the Settings Page (Authentication)}
    \label{tab:tests-settings-auth}
\end{table}

\begin{table}[htp]
    \centering
    \begin{tabular}{c|l|l}
        \textnumero & Test Description                        & Expected Outcome                             \\
        \hline
        1b(i)       & Press WebSocket button                  & WebSocket connected                          \\
        1b(ii)      & Press refresh WebSocket List Button     & List displayed, with empty connections       \\
        1b(iii)     & Press disconnect button for current App & WebSocket disconnected                       \\
        1b(iv)      & Press refresh WebSocket List Button     & Disconnected WebSocket connection disappears \\
        1b(v)       & Press disconnect button for another App & WebSocket disconnected                       \\
        1b(vi)      & Press refresh WebSocket List Button     & Disconnected WebSocket connection disappears \\
    \end{tabular}
    \caption{Tests for the Settings Page (WebSocket)}
    \label{tab:tests-settings-websocket}
\end{table}

\begin{table}[htp]
    \centering
    \begin{tabular}{c|l|l}
        \textnumero & Test Description             & Expected Outcome                                          \\
        \hline
        1c(i)       & Set the language to English  & Most of the App in English                                \\
        1c(ii)      & Close and relaunch the App   & Language settings persists, correct language in whole App \\
        1c(iii)     & Set the language to Japanese & Most of the App in Japanese                               \\
        1c(iv)      & Close and relaunch the App   & Language settings persists, correct language in whole App \\
        1c(v)       & Set the language to Chinese  & Most of the App in Chinese                                \\
        1c(vi)      & Close and relaunch the App   & Language settings persists, correct language in whole App
    \end{tabular}
    \caption{Tests for the Settings Page (Language)}
    \label{tab:tests-settings-language}
\end{table}

The tests for the past-earthquake page are outlined in \autoref{tab:tests-past-earthquake}.

\begin{table}[htp]
    \centering
    \begin{tabular}{c|l|l}
        \textnumero & Test Description                   & Expected Outcome                                   \\
        \hline
        2(i)        & Press load earthquakes             & List loaded and displayed correctly                \\
        2(ii)       & Press load extra earthquakes       & More earthquakes loaded and appended to the list   \\
        2(iii)      & Press load earthquakes             & List re-loaded (with the extra earthquake removed) \\
        2(iv)       & Choose a specific earthquake       & Details displayed on side panel                    \\
        2(v)        & (None)                             & Map with hypocentre on top layer                   \\
        2(vi)       & (None)                             & Map with observation stations on next layer        \\
        2(vii)      & (None)                             & Map with regions on bottom                         \\
        2(viii)     & (None)                             & Map zoomed to regional intensities                 \\
        2(ix)       & (None)                             & Side panel details displayed and formatted         \\
        2(x)        & (None)                             & Intensity tree displayed and formatted             \\
        2(xi)       & Press view Yahoo earthquake button & Webpage to earthquake launched                     \\
        2(xii)      & Another earthquake chosen          & Application behaves as in 2(v) to 2(xi)            \\
        2(xiii)     & Press load earthquakes             & Details and the map reset
    \end{tabular}
    \caption{Tests for the Past Earthquake Page}
    \label{tab:tests-past-earthquake}
\end{table}

The tests for the real-time page is outlined in \autoref{tab:tests-realtime-kmoni} for Kyoshin Monitor, \autoref{tab:tests-realtime-eew} for EEWs, \autoref{tab:tests-realtime-tsunami} for Tsunami Warnings.

\begin{table}[htp]
    \centering
    \begin{tabular}{c|l|l}
        \textnumero & Test Description                           & Expected Outcome                         \\
        \hline
        3a(i)       & (None)                                     & Current time refreshing                  \\
        3a(ii)      & (None)                                     & Kyoshin Monitor data is displayed on map \\
        3a(iii)     & (None)                                     & Scale of colour is displayed correctly   \\
        3a(iv)      & (None)                                     & Quantity on scale is displayed correctly \\
        3a(v)       & (None)                                     & Data of scale is displayed correctly     \\
        3a(vi)      & Change stations settings to borehole       & Less observation station displayed       \\
        3a(vii)     & Relaunch application                       & Settings Persist                         \\
        3a(viii)    & Change stations settings to surface        & More observation station displayed       \\
        3a(ix)      & Relaunch application                       & Settings Persist                         \\
        3a(x)       & Change data settings to PGA                & Data and scale is displayed correctly    \\
        3a(xi)      & Relaunch application                       & Settings Persist                         \\
        3a(xii)     & Change data settings to Measured Intensity & Data and scale is displayed correctly    \\
        3a(xiii)    & Relaunch application                       & Settings Persist
    \end{tabular}
    \caption{Tests for the Real-time Page (Kyoshin Monitor)}
    \label{tab:tests-realtime-kmoni}
\end{table}

\begin{table}[htp]
    \centering
    \begin{tabular}{c|l|l}
        \textnumero & Test Description                           & Expected Outcome                      \\
        \hline
        3b(i)       & (Feed First Mock EEW)                      & EEW details is displayed on sidebar   \\
        3b(ii)      & (None)                                     & Window is switched to real-time page  \\
        3b(iii)     & (None)                                     & Hypocentre marked                     \\
        3b(iv)      & (None)                                     & Wavefronts drawing and updating       \\
        3b(v)       & (None)                                     & Predicted maximum intensity displayed \\
        3b(vi)      & (Feed Second Mock EEW, assumed hypocentre) & EEW displayed on sidebar              \\
        3b(vii)     & (None)                                     & Hypocentre marked (circular symbol)   \\
        3b(viii)    & (None)                                     & Wavefronts not displayed              \\
        3b(ix)      & (None)                                     & Predicted maximum intensity displayed \\
        3b(x)       & (None)                                     & Details panel switching between EEWs  \\
        3b(xi)      & (Feed Updated First Mock EEW)              & EEW Behaving as in 3b(ii) to 3b(v)    \\
        3b(xii)     & (Feed Old First Mock EEW)                  & EEW Information not Updated           \\
        3b(xiii)    & (None)                                     & EEW removed after time
    \end{tabular}
    \caption{Tests for the Real-time Page (EEWs)}
    \label{tab:tests-realtime-eew}
\end{table}

\begin{table}[htp]
    \centering
    \begin{tabular}{c|l|l}
        \textnumero & Test Description            & Expected Outcome                     \\
        \hline
        3c(i)       & (Feed Mock Tsunami Warning) & Tsunami Warning is displayed         \\
        3c(ii)      & (None)                      & Window is switched to real-time page \\
        3c(iii)     & (None)                      & Shorelines coloured and displayed    \\
        3c(iv)      & (None)                      & Tsunami Warning details displayed    \\
        3c(v)       & (None)                      & Tsunami Warning removed after time
    \end{tabular}
    \caption{Tests for the Real-time Page (Tsunamis)}
    \label{tab:tests-realtime-tsunami}
\end{table}

The tests for robustness testing are outlined in \autoref{tab:tests-robustness}.

\begin{table}[htp]
    \centering
    \begin{tabular}{c|l|l}
        \textnumero & Test Description                          & Expected Outcome                   \\
        \hline
        4(i)        & Press buttons when Unauthorised           & Application does not crash         \\
        4(ii)       & Input invalid API Key Format              & Button to Set disabled             \\
        4(iii)      & Set Incorrect API Key                     & Authorisation automatically resets \\
        4(iv)       & Set Invalid OAuth 2 Refresh Token         & Authorisation automatically resets \\
        4(v)        & Unstable Internet Connection              & Application does not crash         \\
        4(vi)       & Clicking buttons at rapid pace            & Application does not crash         \\
        4(vii)      & Loading earthquakes where details unknown & 'Unknown' text displayed
    \end{tabular}
    \caption{Tests for Robustness}
    \label{tab:tests-robustness}
\end{table}

\section{Testing Video}
\begin{itemize}
    \item You can include a video to assist (but you will need to reference the time point at which relevant evidence appears)
    \item If you include a video you will need to have it publicly available.
    \item It is suggested that you include a QR code in your testing to give a link to it the video (for the moderator) rather than just giving a long URL on its own.
\end{itemize}

\section{Testing against Original Requirements' Specification}
