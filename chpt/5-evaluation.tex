\section{Requirements Specification Evaluation}
Personal evaluation
\begin{itemize}
    \item Copy and paste your original requirements from your project analysis
    \item You need to review each requirement and comment objectively on whether it was \textit{fully met/partially met/not met}.
\end{itemize}

\begin{table}[!ht]
    \centering

    \begin{tabular}{|l|p{0.15\linewidth}|l|p{0.3\linewidth}|}
        \hline
        Requirement \textnumero & Description & Success Criteria & Fully/Partial/Not met (Reflective Comment) \\
        \hline \hline
                                &             &                  &                                            \\
        \hline
                                &             &                  &                                            \\
        \hline
                                &             &                  &                                            \\
        \hline
    \end{tabular}
    \caption{Table of Evaluation.}
    \label{table:evaluation}
\end{table}

\section{Independent End-User Feedback}
End user/client evaluation
\begin{itemize}
    \item there \textbf{must} be meaningful end user feedback
    \item You should hold a review meeting with your end user
    \item Write down any key feedback that they give you. E.g. Agreement that a particular requirement has been meet/comments as to aspects that they find suboptimal/comments as to additions they would like to see
\end{itemize}

\begin{table}[!ht]
    \centering

    \begin{tabular}{|l|p{0.15\linewidth}|l|p{0.3\linewidth}|}
        \hline
        Requirement \textnumero & Description & Acceptance Y/N & Additional Comments \\
        \hline \hline
                                &             &                &                     \\
        \hline
                                &             &                &                     \\
        \hline
                                &             &                &                     \\
        \hline
    \end{tabular}
    \caption{Table of Feedback.}
    \label{table:feedback}
\end{table}

\section{Improvements}

You need to give consideration to a number of potential future improvements that could be made. They may arise from either your experience or from feedback given to you by your end user. Ideally at least one should be in response to end user feedback.

\begin{itemize}
    \item Write a paragraph for each potential improvement/change
    \item The improvements/changes could result from additional functionality that has been identified as being beneficial or could be as a result of required efficiencies if some processes are clunky or require faster run-times
    \item You should then comment on how the proposed change could be implemented moving forward. i.e. what would need to be changed/developed and how? You are not expected to actually make any changes; just comment on the possibilities.
\end{itemize}